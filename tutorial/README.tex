\section{coulomb2gmt -- pre-released
v1.0-rc1.0}\label{coulomb2gmt-pre-released-v1.0-rc1.0}

\begin{quote}
Bash scripts to plot coulomb output on GMT
\end{quote}
% 
% \href{https://github.com/demanasta/coulomb2gmt/blob/master/LICENSE}{\includegraphics{http://img.shields.io/badge/license-MIT-brightgreen.svg}}
% \href{https://github.com/coulomb2gmt/pres-templates/releases/latest}{\includegraphics{https://img.shields.io/github/release/demanasta/doulomb2gmt.svg}}
% \href{https://github.com/demanasta/coulomb2gmt/tags}{\includegraphics{https://img.shields.io/github/tag/demanasta/coulomb2gmt.svg}}
% \href{https://github.com/demanasta/coulomb2gmt/stargazers}{\includegraphics{https://img.shields.io/github/stars/demanasta/coulomb2gmt.svg}}
% \href{https://github.com/demanasta/coulomb2gmt/network}{\includegraphics{https://img.shields.io/github/forks/demanasta/coulomb2gmt.svg}}
% \href{https://github.com/demanasta/coulomb2gmt/issues}{\includegraphics{https://img.shields.io/github/issues/demanasta/coulomb2gmt.svg}}

\subsection{Author(s)}\label{authors}

\textbf{Demitris G. Anastasiou}

Rural \& Surveying Engineer, NTUA

Dr.Eng in Geodesy, NTUA

contact:

-- mail: \texttt{\url{dganastasiou@gmail.com}}

-- linkedin: \texttt{\url{https://www.linkedin.com/in/demitrisanastasiou/}}

-- web: \texttt{\url{https://demanasta.github.io}}

\textbf{Decleration}

The present project was developed during my doctoral dissertation, at the Laboratory 
of Higher Geodesy and Dionysos Sastellite Observatory, at the School of Rural \& Surveying
Engineering of National Technical University of Athens.

\subsection{MIT License}

\textbf{Copyright (c) 2017 D. G. Anastasiou}


Permission is hereby granted, free of charge, to any person obtaining a copy
of this software and associated documentation files (the "Software"), to deal
in the Software without restriction, including without limitation the rights
to use, copy, modify, merge, publish, distribute, sublicense, and/or sell
copies of the Software, and to permit persons to whom the Software is
furnished to do so, subject to the following conditions:
\\ \\
The above copyright notice and this permission notice shall be included in all
copies or substantial portions of the Software.
\\ \\
THE SOFTWARE IS PROVIDED "AS IS", WITHOUT WARRANTY OF ANY KIND, EXPRESS OR
IMPLIED, INCLUDING BUT NOT LIMITED TO THE WARRANTIES OF MERCHANTABILITY,
FITNESS FOR A PARTICULAR PURPOSE AND NONINFRINGEMENT. IN NO EVENT SHALL THE
AUTHORS OR COPYRIGHT HOLDERS BE LIABLE FOR ANY CLAIM, DAMAGES OR OTHER
LIABILITY, WHETHER IN AN ACTION OF CONTRACT, TORT OR OTHERWISE, ARISING FROM,
OUT OF OR IN CONNECTION WITH THE SOFTWARE OR THE USE OR OTHER DEALINGS IN THE
SOFTWARE.

\section{Features}\label{features}

\begin{itemize}
\item Auto-configure map lat-long from input files (.inp)
\item
  Plot Stress changes (Coulomb, Normal, Shear)
\item
  Plot cross section for stress changes and dilatation.
\item
  Plot all strain components (E**, Dilatation)
\item
  Overlay stress/strain on the top of topographic DEM.
\item
  Plot Fault geometry (Projection, Surface, Depth).
\item
  Plot GPS displacements observed and modeled.
\item
  Plot Fault and CMT databases and earhtquake distribution.
\item
  Add GMT timestamp logo and custom logo of your organization.
\item
  Adjust paper size to map and convert in different output formats
  (.jpg, .png, .eps, .pdf).
\end{itemize}

\section{Requirements}\label{requirements}

\begin{itemize}
\item
  \textbf{GMT}: \href{http://gmt.soest.hawaii.edu/}{The Generic Mappting
  Tools - GMT} version \textgreater{} 5.1.1 . It is recommented to
  install it from source code.

  \begin{itemize}
  \item
    for \emph{Ubuntu/Debian}: if you use default package installation
    you have to install also \texttt{libgmt-dev} package
  \end{itemize}
\item
  \textbf{Coulomb 3}:
  \href{https://earthquake.usgs.gov/research/software/coulomb/}{Coulomb
  3, developed by USGS}
\item
  \textbf{python}: required for some math calculations included in the
  main script.
\end{itemize}

\section{Usage details}\label{usage-details}

The main script is: \texttt{coulomb2gmt.sh}

run:

\begin{verbatim}
 ./coulomb2gmt.sh  <inputfile> <inputdata> | options
\end{verbatim}

\begin{itemize}
\item
  \texttt{\textless{}inputfile\textgreater{}}: name of input file used
  from Coulomb. Extention \texttt{.inp} not needed. Path to the
  directory of input files configured at \texttt{default-param}.
\item
  \texttt{\textless{}inputdata\textgreater{}}: Code name of input files
  include results of coulmb calculations. Input data files are:
\end{itemize}

\emph{Fault geometry files:}

\begin{itemize}
\item
  \texttt{\textless{}inputdata\textgreater{}-gmt\_fault\_surface.dat}:
  Source and receiver faults' trace at surface.
\item
  \texttt{\textless{}inputdata\textgreater{}-gmt\_fault\_map\_proj.dat}:
  Surface of source and receiver faults.
\item
  \texttt{\textless{}inputdata\textgreater{}-gmt\_fault\_calc\_dep.dat}:
  Intersection of target depth with fault plane.
\end{itemize}

\emph{Stress change output files:}

\begin{itemize}
\item
  \texttt{\textless{}inputdata\textgreater{}-coulomb\_out.dat}: Coulomb
  matrix data output.
\item
  \texttt{\textless{}inputdata\textgreater{}-dcff.cou}: Output of all
  stress components.
\item
  \texttt{\textless{}inputdata\textgreater{}-dcff\_section.cou}: Output
  of all stress components in cross section.
\item
  \texttt{\textless{}inputdata\textgreater{}-Cross\_section.dat}: Cross
  section parameters.
\item
  \texttt{\textless{}inputdata\textgreater{}-Focal\_mech\_stress\_output.csv}:
\end{itemize}

\emph{Strain output files:}

\begin{itemize}

\item
  \texttt{\textless{}inputdata\textgreater{}-Strain.cou}: Data matrix of
  starin components.
\end{itemize}

\emph{Earthquakes, GPS, custom text files:}

\begin{itemize}
\item
  Earthquakes distribution: Earthquakes catalogue files. Structure is

\begin{verbatim}
line1: Header line
line2: Header line
line*: YEAR MONTH DAY    HH MM SS    LAT.   LONG.  DEPTH    MAGNITUDE  (10 fields)
\end{verbatim}
\item
  Centroid Moment Tensors file: Structure of file is the old GMT format
  for CMT. Use \# to comment lines.

\begin{verbatim}
line* :  lon  lat   d  str dip slip str dip slip magnt exp plon  plat  name (14 fields)
\end{verbatim}
\item
  Custom text files: Use new gmt format for \texttt{pstext}. (GMT ver
  \textgreater{} 5.1 )

\begin{verbatim}
line* :lon lat font_size,font_type,font_color angle potision text
\end{verbatim}
\item
  \texttt{\textless{}inputdata\textgreater{}-gps.dist}: GPS
  displacements.
\end{itemize}

\begin{quote}
All paths can be configured in the \texttt{deafault-param} file. Default
the paths are where coulomb create by default each file.
\end{quote}

\subsection{Default parameters}\label{default-parameters}

Many parameters configured at \texttt{default-param} file. 
\begin{enumerate}
\item  Paths togeneral files (DEM, logo, faults) 
\begin{verbatim}
export pth2dems=

export inputTopoL=${pth2dems}/
export inputTopoB=${pth2dems}/

# Path to logo file
export pth2logo=

# Path to faults database file
export pth2faults=
\end{verbatim}
\item Paths to input file directories (.inp, .dat, .cou, .disp)
\begin{verbatim}
# Path to directory of input files *.inp
export pth2inpdir=../input_kefa14/ #../inp_historic/

# Path to directory of *.dat files
export pth2datdir=../gmt_files/

# Path to directory of *.cou files
export pth2coudir=../output_files/

# Path to directory of *.disp files. GPS displacements
export pth2gpsdir=../gps_data/

# Path to earthquakes database
export pth2eqdir=../earthquake_data/

\end{verbatim} 
\item ColorMaps Palette, frame variable. 
\begin{verbatim}
export landcpt=tmp_land.cpt
export bathcpt=tmp_bath.cpt
export coulombcpt=Coulomb_anatolia.cpt
\end{verbatim}
\item General variables.
\begin{verbatim}
# Do not comment next 3 line
export prjscale=1500000         ## Default 1500000
export frame=0.5		## Default 0.5
export sclength=20              ## Default 20

# general variables

# set bar range for stress/strain sale bar
export barrange=1		## Default 1

# Used for strain plot <stain>*10^-${strainscale}
export strainscale=5    ## Default 5

# Horizontal and vertical displacement scale
export dhscale=100		## Default 100
export dvscale=100		## Default 100

# magrnitude of displacement scale legend
export dhscmagn=10 #use mm
export dvscmagn=12 #use mm

# Set calculation depth in km. Default read input file
# export CALC_DEPTH=8 #use km

# set transparency for overlay topography
export RTRANS=30 

# set code of fault name used on input files of coulomb
export FAULT_CODE=Fault

# Set position fron gmt and custom logos
export logogmt_pos="-UBL/0.05c/0.05c/DemAnast"
export logocus_pos="-Dx13.57c/10.05c+w1.1c"
\end{verbatim}
\end{enumerate}

\subsection{General options}\label{general-options}

\begin{itemize}
\item
  \texttt{-r\ \ \ \textbar{}\ -\/-region}: set custom region parameters.
  \emph{Structure} \texttt{-r\ minlon\ maxlon\ minlat\ maxlat\ prjscale}
\item
  \texttt{-t\ \ \ \textbar{}\ -\/-topography}: plot topography using DEM
  file
\item
  \texttt{-o\ \ \textbar{}\ -\/-output\ \textless{}filename\textgreater{}}:
  set custom name of output file. Default is
  \texttt{\textless{}inputdata\textgreater{}}.
\item
  \texttt{-cmt\ \textbar{}\ -\/-moment\_tensor\ \textless{}file\textgreater{}}
  : Plot Centroid Moment Tensors list of earthquakes.
\item
  \texttt{-ed\ \textbar{}\ -\/-eq\_distribution\ \textless{}file\textgreater{}}
  : Plot earthquakes distribution. No classification.
\item
  \texttt{-fl\ \textbar{}\ -\/-faults\_db}: Plot custom fault database
  catalogue.
\item
  \texttt{-mt\ \textbar{}\ -\/-map\_title\ \ "map\ title"}: Custom map
  title.
\item
  \texttt{-ct\ \textbar{}\ -\/-custom\_text\ \ \textless{}path\ to\ file\textgreater{}}
  : Plot Custom text file.
\item
  \texttt{-lg\ \textbar{}\ -\/-logo\_gmt}: Plot GMT logo and time stamp.
\item
  \texttt{-lc\ \textbar{}\ -\/-logo\_custom}: Plot custom logo (image)
  of your organization.
\item
  \texttt{-h\ \textbar{}\ -\/-help}: Help menu.
\item
  \texttt{-v\ \textbar{}\ -\/-version}: Plot version.
\item
  \texttt{-d\ \textbar{}\ -\/-debug}: Enable Debug option.
\end{itemize}

\subsection{Plot fault parameters}\label{plot-fault-parameters}

\begin{itemize}
\item
  \texttt{-fproj}: Plot source and receiver faults' trace at surface.
\item
  \texttt{-fsurf}: Plot surface of source and receiver faults.
\item
  \texttt{-fdep}: Plot intersection of target depth with fault plane.
\end{itemize}

\subsection{Plot stress}\label{plot-stress}

\begin{itemize}
\item
  \texttt{-cstress}: Plot Coulomb Stress change.
\item
  \texttt{-sstress}: Plot Shear Stress change.
\item
  \texttt{-nstress}: Plot Normal Stress change.
\item
  \texttt{-fcross}: Plot cross section of stress change or dilatation.
\end{itemize}

\subsection{Plot Strain components}\label{plot-strain-components}

\begin{itemize}
\item
  \texttt{-stre**}: Where \texttt{**} you can fill all strain components
  \texttt{xx},\texttt{yy},\texttt{zz}, \texttt{yz}, \texttt{xz},
  \texttt{xy}.
\item
  \texttt{-strdil}: Plot dilatation (Exx + Eyy + Ezz )
\end{itemize}

\subsection{Overlay Stress/strain on the top of DEM}

\texttt{-****+ot}: use \texttt{+ot} after the main argument to overlay
the raster output on the top of DEM. configure transparency in
\texttt{default-param} file. \textbf{Be careful} transparency can
printed only in JPEG, PNG and PDF outputs.

\subsection{Plot gps velocities, observed and
modeled}\label{plot-gps-velocities-observed-and-modeled}

\begin{itemize}
\item
  \texttt{-dgpsho}: Observed GPS horizontal displacements.
\item
  \texttt{-dgpshm}: Modeled horizontal displacements on GPS sites (Okada
  1985).
\item
  \texttt{-dgpsvo}: Observed GPS vertical desplacements.
\item
  \texttt{-dgpsvm}: Modeled vertical displacements on GPS sites (Okada
  1985).
\end{itemize}

\begin{quote}
Configure displacement scale in \texttt{default-param} file
\end{quote}

\subsection{Output formats}\label{output-formats}

Default format is \texttt{*.ps} file. You can use the options bellow to
convert to other format and adjust paper size to map size.

\begin{itemize}
\item
  \texttt{-outjpg} : Adjust and convert to JPEG.
\item
  \texttt{-outpng} : Adjust and convert to PNG (transparent where
  nothing is plotted).
\item
  \texttt{-outeps} : Adjust and convert to EPS.
\item
  \texttt{-outpdf} : Adjust and convert to PDF.
\end{itemize}

\section{Move and rename coulomb output
files}\label{move-and-rename-coulomb-output-files}

An assistant script \texttt{mvclbfiles.sh} developed to move and rename
all output files in specific directories on coulomb home directory.

You must first set \texttt{CLB34\_HOME} variable the path to coulomb
home directory, etc.

\texttt{\$\ export\ CLB34\_HOME=\$\{HOME\}/coulomb34}

\textbf{Usage}:
\texttt{\$\ ./mvclsbfiles.sh\ \textless{}inputdata\textgreater{}}

\texttt{\textless{}inputdata\textgreater{}} is the code as mentioned in
the main script above.

\textbf{Files, rename and move:}

\begin{verbatim}
1. `coulomb_out.dat`  
    -> `/gmt_files/<inputdata>-coulomb_out.dat`
\end{verbatim}

\emph{Fault geometry files}

\begin{verbatim}
2. `gmt_fault_calc_dep.dat` 
    -> `/gmt_files/<inputdata>-gmt_fault_calc_dep.dat`

3. `gmt_fault_map_proj.dat` 
    ->  `/gmt_files/<inputdata>-gmt_fault_map_proj.dat`

4. `gmt_fault_surface.dat`  
    -> `/gmt_files/<inputdata>-gmt_fault_surface.dat`
\end{verbatim}

\emph{GPS displacements}

\begin{verbatim}
5. `/output_files/GPS_output.csv` 
    -> ` /gps_data/<inputdata>-gps.disp`
\end{verbatim}

\emph{Stress change files}

\begin{verbatim}
6. `/output_files/Cross_section.dat` .
    ->  `/output_files/<inputdata>-Cross_section.dat`

7. `/output_files/dcff.cou` 
    ->  `/output_files/<inputdata>-dcff.cou`

8. `/output_files/dcff_section.cou` 
    -> `/output_files/<inputdata>-dcff_section.cou`

9. `/output_files/dilatation_section.cou` 
    -> `/output_files/<inputdata>-dilatation_section.cou`

10. `/output_files/Focal_mech_stress_output.csv` 
    -> `/output_files/<inputdata>-Focal_mech_stress_output.csv`
\end{verbatim}

\emph{Strain output files}

\begin{verbatim}
11. `/output_files/Strain.cou` 
    -> `/output_files/<inputdata>-Strain.cou`
\end{verbatim}

\section{Contributing}\label{contributing}

\begin{enumerate}
\def\labelenumi{\arabic{enumi}.}

\item
  Create an issue and describe your idea
\item
  \href{https://github.com/demanasta/coulomb2gmt/network\#fork-destination-box}{Fork
  it}
\item
  Create your feature branch (\texttt{git\ checkout\ -b\ my-new-idea})
\item
  Commit your changes
  (\texttt{git\ commit\ -am\  Add\ some\ feature})
\item
  Publish the branch (\texttt{git\ push\ origin\ my-new-idea})
\item
  Create a new Pull Request
\item
  Profit!
\end{enumerate}

\subsection{For contributors}\label{for-contributors}

\subsubsection{git structure}\label{git-structure}

\begin{itemize}

\item
  \textbf{branch} master/develop: stucture of main branch

  \begin{itemize}
  
  \item
    bash scripts

    \begin{itemize}
    
    \item
      coulomb2gmt.sh: main script
    \item
      mvclbfiles.sh: assistant script, move and rename files
    \item
      default-param: configure parameters
    \end{itemize}
  \item
    functions: bash functions called from main script

    \begin{itemize}
    
    \item
      messages.sh: help function and ptint messages.
    \item
      gen\_func.sh: general functions.
    \item
      clbplots.sh: functions for gmt plots
    \item
      checknum.sh: check number function.
    \end{itemize}
  \item
    docs: MarkDown templates for issues, pull requests, contributions
    etc.
  \end{itemize}
\item
  \textbf{branch} documents :

  \begin{itemize}
  
  \item
    tutorial: reference and user guide, tex files.
  \item
    examples: presentation of examples, tex/beamer files.
  \end{itemize}
\item
  \textbf{branch} testcase: Include configured files for testing the
  script.
\end{itemize}

\subsubsection{Simple guidlines for
coding}\label{simple-guidances-for-coding}

\begin{itemize}

\item
  \textbf{general}

  \begin{itemize}
  
  \item
    Use 80 characters long line.
  \item
    Surround variables with \texttt{\{\}} and use \texttt{"\$\{\}"} in
    \texttt{if} case.
  \item
    Use comments in coding.
  \end{itemize}
\item
  \textbf{gmt functions}

  \begin{itemize}
  
  \item
    Use \texttt{-K} \texttt{-O} \texttt{-V\$\{VRBLEVM\}} at the end of
    each function.
  \item
    Create a function if a part of the script will be used more than two
    times.
  \item
    Add printed comments and debug messages in the code.
  \end{itemize}
\end{itemize}

\section{ChangeLog}\label{changelog}

The history of releases can be viewed at
\href{docs/ChangeLog.md}{ChangeLog.md}

\section{Acknowlegments}\label{acknowlegments}

\section{References}\label{references}

\begin{itemize}
\item
  \href{https://earthquake.usgs.gov/research/software/coulomb/}{Coulomb
  3, developed by USGS}
\item
  Toda, Shinji, Stein, R.S., Sevilgen, Volkan, and Lin, Jian, 2011,
  Coulomb 3.3 Graphic-rich deformation and stress-change software for
  earthquake, tectonic, and volcano research and teaching---user guide:
  U.S. Geological Survey Open-File Report 2011-1060, 63 p., available at
  http://pubs.usgs.gov/of/2011/1060/.
\item
  \href{http://gmt.soest.hawaii.edu/}{The Generic Mapping Tools - GMT}
\item
  Python Software Foundation. Python Language Reference, version 2.7.
  Available at http://www.python.org
\end{itemize}
