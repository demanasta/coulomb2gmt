\section[Intro]{Introduction}
 
% \graphicspath{Figs/}

\begin{frame}\frametitle{\texttt{coulomb2gmt project}}\framesubtitle{}
\begin{quote}
Bash scripts to plot coulomb output on GMT
\end{quote}
\vskip -1cm
\begin{columns}
  \begin{column}{.5\textwidth}
\textbf{Demitris G. Anastasiou}

\begin{footnotesize}
Rural \& Surveying Engineer, NTUA

Dr.Eng in Geodesy, NTUA
\end{footnotesize}
\vfill
\textbf{Decleration}

\begin{footnotesize}
The present project was developed during my doctoral dissertation, at the Laboratory 
of Higher Geodesy and Dionysos Sastellite Observatory, at the School of Rural \& Surveying
Engineering of National Technical University of Athens.
\end{footnotesize}
  \end{column}
  \begin{column}{.5\textwidth}
\begin{tiny}

\begin{flushright}
\textbf{MIT License}

\textbf{Copyright (c) 2017\\D. G. Anastasiou}
\end{flushright}


%Permission is hereby granted, free of charge, to any person obtaining a copy
%of this software and associated documentation files (the "Software"), to deal
%in the Software without restriction, including without limitation the rights
%to use, copy, modify, merge, publish, distribute, sublicense, and/or sell
%copies of the Software, and to permit persons to whom the Software is
%furnished to do so, subject to the following conditions:
%\\
%The above copyright notice and this permission notice shall be included in all
%copies or substantial portions of the Software.
%\\
THE SOFTWARE IS PROVIDED "AS IS", WITHOUT WARRANTY OF ANY KIND, EXPRESS OR
IMPLIED, INCLUDING BUT NOT LIMITED TO THE WARRANTIES OF MERCHANTABILITY,
FITNESS FOR A PARTICULAR PURPOSE AND NONINFRINGEMENT. IN NO EVENT SHALL THE
AUTHORS OR COPYRIGHT HOLDERS BE LIABLE FOR ANY CLAIM, DAMAGES OR OTHER
LIABILITY, WHETHER IN AN ACTION OF CONTRACT, TORT OR OTHERWISE, ARISING FROM,
OUT OF OR IN CONNECTION WITH THE SOFTWARE OR THE USE OR OTHER DEALINGS IN THE
SOFTWARE.
\end{tiny}
  \end{column}
\end{columns}

\end{frame}

\begin{frame}
\frametitle{\texttt{coulomb2gmt project:} Features}


\begin{itemize}
\item Auto-configure map lat-long from input files (.inp)
\item
  Plot Stress changes (Coulomb, Normal, Shear)
\item
  Plot cross section for stress changes and dilatation.
\item
  Plot all strain components (E**, Dilatation)
\item
  Overlay stress/strain on the top of topographic DEM.
\item
  Plot Fault geometry (Projection, Surface, Depth).
\item
  Plot GPS displacements observed and modeled.
\item
  Plot Fault and CMT databases and earhtquake distribution.
\item
  Add GMT timestamp logo and custom logo of your organization.
\item
  Adjust paper size to map and convert in different output formats
  (.jpg, .png, .eps, .pdf).
\end{itemize}

\end{frame}

\begin{frame}
\frametitle{\texttt{coulomb2gmt project:} Requirements}

\begin{itemize}
\item
  \textbf{GMT}: \href{http://gmt.soest.hawaii.edu/}{The Generic Mappting
  Tools - GMT} version \textgreater{} 5.1.1 . It is recommented to
  install it from source code.

  \begin{itemize}
  \item
    for \emph{Ubuntu/Debian}: if you use default package installation
    you have to install also \texttt{libgmt-dev} package
  \end{itemize}
\item
  \textbf{Coulomb 3}:
  \href{https://earthquake.usgs.gov/research/software/coulomb/}{Coulomb
  3, developed by USGS}
\item
  \textbf{python}: required for some math calculations included in the
  main script.
\end{itemize}
\end{frame}

