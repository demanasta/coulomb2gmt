\section[Inputs]{Input Files}
 
\graphicspath{Chapter2/Figs/}

% //////////////////////////////////////////////////////////////////////////////
\begin{frame}[t,fragile]
  \frametitle{Input files}
  \framesubtitle{}
  \label{ch2fr:inputclb}
The main script is: \texttt{coulomb2gmt.sh}

run:

\begin{verbatim}
$ ./coulomb2gmt.sh <inputfile> <inputdata> | options
\end{verbatim}

\begin{itemize}
\item
  \texttt{\textless{}inputfile\textgreater{}}: name of input file used
  from Coulomb. Extention \texttt{.inp} not needed. Path to the
  directory of input files configured at \texttt{default-param}.
\item
  \texttt{\textless{}inputdata\textgreater{}}: Code name of input files
  include results of coulmb calculations. Input data files are:
\end{itemize}
\end{frame}
\note{} % Add notes for this slide

\begin{frame}[t,fragile]
  \frametitle{Input files}
  \framesubtitle{}
  \label{ch2fr:inputclb2}
\emph{Fault geometry files:}

\begin{itemize}
\item
  \texttt{\textless{}inputdata\textgreater{}-gmt\_fault\_surface.dat}:
  Source and receiver faults' trace at surface.
\item
  \texttt{\textless{}inputdata\textgreater{}-gmt\_fault\_map\_proj.dat}:
  Surface of source and receiver faults.
\item
  \texttt{\textless{}inputdata\textgreater{}-gmt\_fault\_calc\_dep.dat}:
  Intersection of target depth with fault plane.
\end{itemize}

\emph{Stress change output files:}

\begin{itemize}
\item
  \texttt{\textless{}inputdata\textgreater{}-coulomb\_out.dat}: Coulomb
  matrix data output.
\item
  \texttt{\textless{}inputdata\textgreater{}-dcff.cou}: Output of all
  stress components.
\item
  \texttt{\textless{}inputdata\textgreater{}-dcff\_section.cou}: Output
  of all stress components in cross section.
\item
  \texttt{\textless{}inputdata\textgreater{}-Cross\_section.dat}: Cross
  section parameters.
\item
  \texttt{\textless{}inputdata\textgreater{}-Focal\_mech\_stress\_output.csv}:
\end{itemize}

\emph{Strain output files:}

\begin{itemize}

\item
  \texttt{\textless{}inputdata\textgreater{}-Strain.cou}: Data matrix of
  starin components.
\end{itemize}

% //////////////////////////////////////////////////////////////////////////////
\end{frame}
\note{} % Add notes for this slide


\begin{frame}[t,fragile]
  \frametitle{Input files}
  \framesubtitle{}
  \label{ch2fr:inputgen}
\emph{Earthquakes, GPS, custom text files:}

\begin{itemize}
\item
  Earthquakes distribution: Earthquakes catalogue files. Structure is

\begin{verbatim}
line1: Header line
line2: Header line
line*: YEAR MONTH DAY HH MM SS LAT. LONG. DEPTH MAGNITUDE  (10f)
\end{verbatim}
\item
  Centroid Moment Tensors file: Structure of file is the old GMT format
  for CMT. Use \# to comment lines.

\begin{verbatim}
line* : lon lat d str dip slip str dip slip magnt exp plon  plat  name (14f)
\end{verbatim}
\item
  Custom text files: Use new gmt format for \texttt{pstext}. (GMT ver
  \textgreater{} 5.1 )

\begin{verbatim}
line* :lon lat font\_size,font\_type,font\_color angle potision text
\end{verbatim}
\item
  \texttt{\textless{}inputdata\textgreater{}-gps.dist}: GPS
  displacements.
\end{itemize}

\begin{quote}
All paths can be configured in the \texttt{default-param} file. Default
the paths are where coulomb create by default each file.
\end{quote}

\end{frame}
\note{} % Add notes for this slide














